\PassOptionsToPackage{unicode=true}{hyperref} % options for packages loaded elsewhere
\PassOptionsToPackage{hyphens}{url}
\PassOptionsToPackage{dvipsnames,svgnames*,x11names*}{xcolor}
%
\documentclass[12pt,a4paper]{article}
\usepackage{lmodern}
\usepackage{amssymb,amsmath}
\usepackage{ifxetex,ifluatex}
\usepackage{fixltx2e} % provides \textsubscript
\ifnum 0\ifxetex 1\fi\ifluatex 1\fi=0 % if pdftex
  \usepackage[T1]{fontenc}
  \usepackage[utf8]{inputenc}
  \usepackage{textcomp} % provides euro and other symbols
\else % if luatex or xelatex
  \usepackage{unicode-math}
  \defaultfontfeatures{Ligatures=TeX,Scale=MatchLowercase}
\fi
% use upquote if available, for straight quotes in verbatim environments
\IfFileExists{upquote.sty}{\usepackage{upquote}}{}
% use microtype if available
\IfFileExists{microtype.sty}{%
\usepackage[]{microtype}
\UseMicrotypeSet[protrusion]{basicmath} % disable protrusion for tt fonts
}{}
\IfFileExists{parskip.sty}{%
\usepackage{parskip}
}{% else
\setlength{\parindent}{0pt}
\setlength{\parskip}{6pt plus 2pt minus 1pt}
}
\usepackage{xcolor}
\usepackage{hyperref}
\hypersetup{
            pdftitle={Notes on calculating combined GAM estimates within the Rapid Assessment Method (RAM)},
            colorlinks=true,
            linkcolor=blue,
            filecolor=Maroon,
            citecolor=blue,
            urlcolor=blue,
            breaklinks=true}
\urlstyle{same}  % don't use monospace font for urls
\usepackage[margin=2cm]{geometry}
\usepackage{color}
\usepackage{fancyvrb}
\newcommand{\VerbBar}{|}
\newcommand{\VERB}{\Verb[commandchars=\\\{\}]}
\DefineVerbatimEnvironment{Highlighting}{Verbatim}{commandchars=\\\{\}}
% Add ',fontsize=\small' for more characters per line
\usepackage{framed}
\definecolor{shadecolor}{RGB}{248,248,248}
\newenvironment{Shaded}{\begin{snugshade}}{\end{snugshade}}
\newcommand{\AlertTok}[1]{\textcolor[rgb]{0.94,0.16,0.16}{#1}}
\newcommand{\AnnotationTok}[1]{\textcolor[rgb]{0.56,0.35,0.01}{\textbf{\textit{#1}}}}
\newcommand{\AttributeTok}[1]{\textcolor[rgb]{0.77,0.63,0.00}{#1}}
\newcommand{\BaseNTok}[1]{\textcolor[rgb]{0.00,0.00,0.81}{#1}}
\newcommand{\BuiltInTok}[1]{#1}
\newcommand{\CharTok}[1]{\textcolor[rgb]{0.31,0.60,0.02}{#1}}
\newcommand{\CommentTok}[1]{\textcolor[rgb]{0.56,0.35,0.01}{\textit{#1}}}
\newcommand{\CommentVarTok}[1]{\textcolor[rgb]{0.56,0.35,0.01}{\textbf{\textit{#1}}}}
\newcommand{\ConstantTok}[1]{\textcolor[rgb]{0.00,0.00,0.00}{#1}}
\newcommand{\ControlFlowTok}[1]{\textcolor[rgb]{0.13,0.29,0.53}{\textbf{#1}}}
\newcommand{\DataTypeTok}[1]{\textcolor[rgb]{0.13,0.29,0.53}{#1}}
\newcommand{\DecValTok}[1]{\textcolor[rgb]{0.00,0.00,0.81}{#1}}
\newcommand{\DocumentationTok}[1]{\textcolor[rgb]{0.56,0.35,0.01}{\textbf{\textit{#1}}}}
\newcommand{\ErrorTok}[1]{\textcolor[rgb]{0.64,0.00,0.00}{\textbf{#1}}}
\newcommand{\ExtensionTok}[1]{#1}
\newcommand{\FloatTok}[1]{\textcolor[rgb]{0.00,0.00,0.81}{#1}}
\newcommand{\FunctionTok}[1]{\textcolor[rgb]{0.00,0.00,0.00}{#1}}
\newcommand{\ImportTok}[1]{#1}
\newcommand{\InformationTok}[1]{\textcolor[rgb]{0.56,0.35,0.01}{\textbf{\textit{#1}}}}
\newcommand{\KeywordTok}[1]{\textcolor[rgb]{0.13,0.29,0.53}{\textbf{#1}}}
\newcommand{\NormalTok}[1]{#1}
\newcommand{\OperatorTok}[1]{\textcolor[rgb]{0.81,0.36,0.00}{\textbf{#1}}}
\newcommand{\OtherTok}[1]{\textcolor[rgb]{0.56,0.35,0.01}{#1}}
\newcommand{\PreprocessorTok}[1]{\textcolor[rgb]{0.56,0.35,0.01}{\textit{#1}}}
\newcommand{\RegionMarkerTok}[1]{#1}
\newcommand{\SpecialCharTok}[1]{\textcolor[rgb]{0.00,0.00,0.00}{#1}}
\newcommand{\SpecialStringTok}[1]{\textcolor[rgb]{0.31,0.60,0.02}{#1}}
\newcommand{\StringTok}[1]{\textcolor[rgb]{0.31,0.60,0.02}{#1}}
\newcommand{\VariableTok}[1]{\textcolor[rgb]{0.00,0.00,0.00}{#1}}
\newcommand{\VerbatimStringTok}[1]{\textcolor[rgb]{0.31,0.60,0.02}{#1}}
\newcommand{\WarningTok}[1]{\textcolor[rgb]{0.56,0.35,0.01}{\textbf{\textit{#1}}}}
\usepackage{longtable,booktabs}
% Fix footnotes in tables (requires footnote package)
\IfFileExists{footnote.sty}{\usepackage{footnote}\makesavenoteenv{longtable}}{}
\usepackage{graphicx,grffile}
\makeatletter
\def\maxwidth{\ifdim\Gin@nat@width>\linewidth\linewidth\else\Gin@nat@width\fi}
\def\maxheight{\ifdim\Gin@nat@height>\textheight\textheight\else\Gin@nat@height\fi}
\makeatother
% Scale images if necessary, so that they will not overflow the page
% margins by default, and it is still possible to overwrite the defaults
% using explicit options in \includegraphics[width, height, ...]{}
\setkeys{Gin}{width=\maxwidth,height=\maxheight,keepaspectratio}
\setlength{\emergencystretch}{3em}  % prevent overfull lines
\providecommand{\tightlist}{%
  \setlength{\itemsep}{0pt}\setlength{\parskip}{0pt}}
\setcounter{secnumdepth}{5}
% Redefines (sub)paragraphs to behave more like sections
\ifx\paragraph\undefined\else
\let\oldparagraph\paragraph
\renewcommand{\paragraph}[1]{\oldparagraph{#1}\mbox{}}
\fi
\ifx\subparagraph\undefined\else
\let\oldsubparagraph\subparagraph
\renewcommand{\subparagraph}[1]{\oldsubparagraph{#1}\mbox{}}
\fi

% set default figure placement to htbp
\makeatletter
\def\fps@figure{htbp}
\makeatother

\usepackage{booktabs}
\usepackage{array}
\usepackage{multirow}
\usepackage{wrapfig}
\usepackage{colortbl}
\usepackage{pdflscape}
\usepackage{tabu}
\usepackage{threeparttable}
\usepackage{threeparttablex}
\usepackage[normalem]{ulem}
\usepackage{makecell}
\usepackage{float}
\usepackage{setspace}
\usepackage{longtable}

\onehalfspacing
\graphicspath{ {figures/} }
\pagenumbering{gobble}
\usepackage{booktabs}
\usepackage{longtable}
\usepackage{array}
\usepackage{multirow}
\usepackage{wrapfig}
\usepackage{float}
\usepackage{colortbl}
\usepackage{pdflscape}
\usepackage{tabu}
\usepackage{threeparttable}
\usepackage{threeparttablex}
\usepackage[normalem]{ulem}
\usepackage{makecell}
\usepackage{xcolor}
\usepackage[]{natbib}
\bibliographystyle{plainnat}

\title{Notes on calculating combined GAM estimates within the Rapid Assessment Method (RAM)}
\author{}
\date{\vspace{-2.5em}19 February 2020}

\begin{document}
\maketitle

{
\hypersetup{linkcolor=}
\setcounter{tocdepth}{3}
\tableofcontents
}
\newpage

\hypertarget{background}{%
\section{Background}\label{background}}

\hypertarget{possible-approach}{%
\section{Possible approach}\label{possible-approach}}

PROBIT gives a probability so we look to combining two probabilities:

\[ P(GAM_{\text{MUAC}} ~ \cup ~ GAM_{\text{WHZ}}) ~ = ~ P(GAM_{\text{MUAC}}) ~ + ~ P(GAM_{\text{WHZ}}) \]

However, the problem is that we do not have \texttt{independent} probabilities. We overestimate because the intersection gets counted twice. Therefore we need:

\[ P(GAM_{\text{MUAC}} ~ \cup ~ GAM_{\text{WHZ}}) ~ = ~ P(GAM_{\text{MUAC}}) ~ + ~ P(GAM_{\text{WHZ}}) ~ - ~ P(GAM_{\text{MUAC}} ~ \cap ~ GAM_{\text{WHZ}}) \]

We have the first two terms but not the third. We can estimate the third term from a 2 by 2 table:

\begin{table}[H]
\centering\begingroup\fontsize{14}{16}\selectfont

\begin{tabular}{>{\ttfamily}l>{\ttfamily}c>{\ttfamily}c}
\toprule
\textbf{ } & \textbf{WHZ $<$ -2} & \textbf{WHZ $\geq$ -2}\\
\midrule
\rowcolor{gray!6}  \ttfamily{MUAC $<$ 125} & \ttfamily{a} & \ttfamily{b}\\
\ttfamily{MUAC $\geq$ 125} & \ttfamily{c} & \ttfamily{d}\\
\bottomrule
\end{tabular}
\endgroup{}
\end{table}

and

\[ P(GAM_{\text{MUAC}} ~ \cap ~ GAM_{\text{WHZ}}) ~ = ~ \frac{a}{a ~ + ~ b ~ + ~ c ~ + ~ d} \]

We have a small sample size so the estimate will lack precision but I think that being ``clever'' and using something like:

\[ P(GAM_{\text{MUAC}} ~ \cap ~ GAM_{\text{WHZ}}) ~ = ~ P(GAM_{\text{MUAC}}) ~ \times ~ P(GAM_{\text{WHZ}}) \]

will not work as it assumes independence. We can try to move forward with this hybrid method.

\newpage

We try this in R using a dataset from Uganda.

\begin{Shaded}
\begin{Highlighting}[]
\CommentTok{## Read dataset}
\NormalTok{x <-}\StringTok{ }\KeywordTok{read.table}\NormalTok{(}\DataTypeTok{file =} \StringTok{"data/ugan01.csv"}\NormalTok{, }\DataTypeTok{header =} \OtherTok{TRUE}\NormalTok{, }\DataTypeTok{sep =} \StringTok{","}\NormalTok{)}

\CommentTok{## Case definitions}
\NormalTok{x}\OperatorTok{$}\NormalTok{gamWHZ <-}\StringTok{ }\KeywordTok{ifelse}\NormalTok{(x}\OperatorTok{$}\NormalTok{whz }\OperatorTok{<}\StringTok{ }\DecValTok{-2}\NormalTok{, }\DecValTok{1}\NormalTok{, }\DecValTok{2}\NormalTok{)}
\NormalTok{x}\OperatorTok{$}\NormalTok{gamMUAC <-}\StringTok{ }\KeywordTok{ifelse}\NormalTok{(x}\OperatorTok{$}\NormalTok{muac }\OperatorTok{<}\StringTok{ }\DecValTok{125}\NormalTok{, }\DecValTok{1}\NormalTok{, }\DecValTok{2}\NormalTok{)}
\NormalTok{x}\OperatorTok{$}\NormalTok{cGAM <-}\StringTok{ }\KeywordTok{ifelse}\NormalTok{(x}\OperatorTok{$}\NormalTok{whz }\OperatorTok{<}\StringTok{ }\DecValTok{-2} \OperatorTok{|}\StringTok{ }\NormalTok{x}\OperatorTok{$}\NormalTok{muac }\OperatorTok{<}\StringTok{ }\DecValTok{125}\NormalTok{, }\DecValTok{1}\NormalTok{, }\DecValTok{2}\NormalTok{)}

\CommentTok{## Classic prevalence}
\KeywordTok{round}\NormalTok{(}\KeywordTok{prop.table}\NormalTok{(}\KeywordTok{table}\NormalTok{(x}\OperatorTok{$}\NormalTok{gamMUAC))[}\DecValTok{1}\NormalTok{] }\OperatorTok{*}\StringTok{ }\DecValTok{100}\NormalTok{, }\DecValTok{2}\NormalTok{)}
\end{Highlighting}
\end{Shaded}

\begin{verbatim}
##    1 
## 13.8
\end{verbatim}

\begin{Shaded}
\begin{Highlighting}[]
\KeywordTok{round}\NormalTok{(}\KeywordTok{prop.table}\NormalTok{(}\KeywordTok{table}\NormalTok{(x}\OperatorTok{$}\NormalTok{gamWHZ))[}\DecValTok{1}\NormalTok{] }\OperatorTok{*}\StringTok{ }\DecValTok{100}\NormalTok{, }\DecValTok{2}\NormalTok{)}
\end{Highlighting}
\end{Shaded}

\begin{verbatim}
##    1 
## 9.05
\end{verbatim}

\begin{Shaded}
\begin{Highlighting}[]
\KeywordTok{round}\NormalTok{(}\KeywordTok{prop.table}\NormalTok{(}\KeywordTok{table}\NormalTok{(x}\OperatorTok{$}\NormalTok{cGAM))[}\DecValTok{1}\NormalTok{] }\OperatorTok{*}\StringTok{ }\DecValTok{100}\NormalTok{, }\DecValTok{2}\NormalTok{)}
\end{Highlighting}
\end{Shaded}

\begin{verbatim}
##    1 
## 15.5
\end{verbatim}

\begin{Shaded}
\begin{Highlighting}[]
\CommentTok{## Test if the two case definitions are independent}
\KeywordTok{chisq.test}\NormalTok{(}\KeywordTok{table}\NormalTok{(x}\OperatorTok{$}\NormalTok{gamMUAC, x}\OperatorTok{$}\NormalTok{gamWHZ))}\OperatorTok{$}\NormalTok{p.value}
\end{Highlighting}
\end{Shaded}

\begin{verbatim}
## [1] 8.810836e-74
\end{verbatim}

\begin{Shaded}
\begin{Highlighting}[]
\CommentTok{## Simple PROBIT prevalence}
\NormalTok{pMUAC <-}\StringTok{ }\KeywordTok{pnorm}\NormalTok{(}\DecValTok{125}\NormalTok{, }\KeywordTok{mean}\NormalTok{(x}\OperatorTok{$}\NormalTok{muac), }\KeywordTok{sd}\NormalTok{(x}\OperatorTok{$}\NormalTok{muac))}
\NormalTok{pWHZ <-}\StringTok{ }\KeywordTok{pnorm}\NormalTok{(}\OperatorTok{-}\DecValTok{2}\NormalTok{, }\KeywordTok{mean}\NormalTok{(x}\OperatorTok{$}\NormalTok{whz), }\KeywordTok{sd}\NormalTok{(x}\OperatorTok{$}\NormalTok{whz))}

\CommentTok{## Estimate the UNION probability}
\NormalTok{pUNION <-}\StringTok{ }\KeywordTok{table}\NormalTok{(x}\OperatorTok{$}\NormalTok{gamMUAC, x}\OperatorTok{$}\NormalTok{gamWHZ)[}\DecValTok{1}\NormalTok{,}\DecValTok{1}\NormalTok{] }\OperatorTok{/}\StringTok{ }\KeywordTok{sum}\NormalTok{(}\KeywordTok{table}\NormalTok{(x}\OperatorTok{$}\NormalTok{gamMUAC, x}\OperatorTok{$}\NormalTok{gamWHZ))}

\CommentTok{## cGAM by PROBIT}
\KeywordTok{round}\NormalTok{((pMUAC }\OperatorTok{+}\StringTok{ }\NormalTok{pWHZ }\OperatorTok{-}\StringTok{ }\NormalTok{pUNION) }\OperatorTok{*}\StringTok{ }\DecValTok{100}\NormalTok{, }\DecValTok{2}\NormalTok{)}
\end{Highlighting}
\end{Shaded}

\begin{verbatim}
## [1] 13.91
\end{verbatim}

\bibliography{bibliography.bib}

\end{document}
